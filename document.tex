\documentclass[runningheads]{llncs}

\usepackage[utf8]{inputenc}
\usepackage[ngerman, english]{babel}

\usepackage[T1]{fontenc}
\usepackage{amssymb}
\setcounter{tocdepth}{3}
\usepackage{graphicx}
\usepackage{amsmath}
\usepackage{mathcomp}
\usepackage{url}
\usepackage[chapter]{algorithm}
\usepackage{algorithmic}
\usepackage{subfigure}

\usepackage{appendix}

\usepackage{listings}
\lstset{numbers=left, numberstyle=\tiny, numbersep=5pt}

\begin{document}
\mainmatter
\title{Trends and Concepts in the Software Industry II: \\ Development of Enterprise Software}
\author{Team 5: Providing Data Context During Development}
\institute{Thomas B\"unger, Felix Leupold, Johan Uhle, \\ Patrick Schilf, Lauritz Thamsen, Fabian Tschirschnitz \\[0.1in] Johannes Wust \\
Franziska Haeger, Dr. Anja Bog, Dr. Juergen M\"uller, Prof. Hasso Plattner \\
Hasso-Plattner Institute \\[0.1in]
March 15, 2013}
\date{March 15th, 2013}
\maketitle

\newpage

%Head of Documentation - Patrick: gerade ziehen

\section{Introduction} 
%HoD
%Seminar, Seminarablauf, Thema (= deren how might we question)
%(dabei Bezug zu unserem Subthema aufbauen)
% close with an outline of this document, include references to all sections, see lables

\section{Design Thinking} \label{sec:DESIGN_THINKING}
%Felix
%intro to Design Thinking, its cycle and some intro to our results 

\section{Understand} \label{sec:UNDERSTAND}
%Johan
%Pre seminar interviews

\section{Observe} \label{sec:OBSERVE}
%Johan
%Bachelor projects, conventional development (edit, build, compile)

\section{Point of View} \label{sec:POINT_OF_VIEW}
%Felix
%Persona
%Problem Statement, our how might we

\section{Ideation} \label{sec:IDEATION}
%Felix
%our idea, plus make up some ideas that were not followed

\section{Paper Prototype} \label{sec:PAPER_PROTOTYPE}
%Thomas

\section{User Testing} \label{sec:USER_TESTING}
%Thomas
%Testing
%Feedback

\section{Final Prototype} \label{sec:FINAL_PROTOTYPE}
%Fabi

\section{Implementation Ideas} \label{sec:IMPLEMENTATION_IDEAS}
%Lauritz
%tracing data contexts = database and parameters and state
%sampling and domain experts

\section{Open Questions}  \label{sec:OPEN_QUESTIONS}
%Lauritz
\subsection{Hardware Context}
\subsection{Query Debugging}
\subsection{Efficient Context Storing}
%HANA time travel

\section{Conclusion} \label{sec:CONCLUSION}
%HoD

%\bibliography{references}
\bibstyle{splncs03}
\bibliographystyle{splncs03}

\end{document}
