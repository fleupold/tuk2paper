%!TEX root = ../document.tex
During our research, we interviewed various people in sofware companies. In this section we will summarize the interviews and the insights we gained from them.

\subsection{Interviews}
\subsubsection{Danny Tramnitzke from Psipenta}
The company \emph{Psipenta} builds an ERP system for SMBs with a focus on manufactoring and suppy chain management. Their core product was developed initially in Cobol and parts have later been and still being ported to C++ and Java. Development happens largely in Eclipse, which has been largly extended and customized by Psipenta with automated code generation and integration into their workflow and environment. Work is flexible with home and remote work.

Our interview partner Danny is working on porting parts of the old code from Cobol to Java as well as developing entirely new features.

Danny estimates the time spent in daily communication at around 30\%. This includes answering emails and attending meetings with coleagues.

In the interview, Danny put an emphasis on code quality. A core means to achieve this was automated testing, best done with involving the customer directly.

As major churns, Danny mentioned interruptions, from co-workers as well as from a slow and/or unintuitive tool chain. As example he mentioned that building a running version of his product to manually test it takes more than half an hour.

\subsubsection{Nicole von Steht from SAP}
SAP is one of the biggest enterprise application companies in the world. Nicole works as Frontend application developer at SAP. Her team consists of 12 people and uses the Scrum methodology. The team is located on one floor and most communication happens face-to-face.

Code quality is a core value, that is assured by unit and integration testing with continuous integration as well as rotating pair programming. Nicole writes code in a customized version of Eclipse.

\subsubsection{Tillman Giese from SAP}
Tillman is working at SAP on the HANA Security Engine. His team consists of 8 people and is split over 2 offices.

Tillman regularly deals with other teams globally distributed, which is often difficult because of time zone differences. He is getting a lot of support requests, which he said most of the time are answered questions that the rquester could have been able to resolve themselves, if they would have just put in effort to research the question correctly. Often, the requests interrupt his workflow. To illustrate this state, Tillman said: \emph{I wish I could work on one task for a full hour}.

As a side project, Tillman is running Agile development workshops at SAP.

\subsubsection{Frank Brunswig from SAP}
Frank is working as Chief Architect at SAP. He's been involved with the company fro 20 years already, worked on various projects and currently oversees the creation of the AppBuilder platform, which enables the creation of custom mobile applications.

Apart from doing small prototype implementations to evaluate a new technology, Frank does not code regularly. His work mainly happenes on a organizational level, where he performs requirement anaysis, organizes his teams, maintains the cross-organizsation-alignment and finds cross-team synergies.

For coding and in his teams Frank prefers all-in-one IDE solutions, which in the best case integrate the complete workflow from finding the requirements over writing code to the final deployment.

His major problem in development are unclear or changing requirements. He is also not an advocate of Agile Methodology, because they badly distribute responsibilities and make the development process harder to control.

\subsubsection{Christof Dorner from Readmill}
The Berlin-based startup Readmill that builds an ebook reader and social network around that. Christof is Backend Developer for the Ruby on Rails application. He works in a development team of 4. The team works with the Kanban methodology.

Christof mentioned an interrupted workflow as an important goal for his day-to-day work. To minimize interruptions through his co-workers, they set up an asynchronous company chat where everyone can participate when they feel like it. Also Christof always works with noise-canceling headphones on. He optimized his tools highly to reduce any wasted waiting time while working. This includes customizing his editor and writing small scripts for churn tasks. Furthermore when choosing the technology for Readmill, the whole team takes into account how easy development with this technology will be and makes developer satisfaction an important component of the decision process.

Tests are a core component of Christof's workflow for developing new features or fixing bugs. He does not do straight test-driven development but rather switches between writing code, testing on the console or web app and writing tests as he likes. He puts an emphasis on the fact that all deployed code has a high test coverage. Code review are done via Pull Requests on Github.

As problems while developing, Christof mentioned refactoring of old code (especially tests), reproducing bugs and interruptions of any kind.

\subsubsection{Ralf Tomczak, Andr\'e Peglow, Holger Hammel and OlliXXXXXXXXXX from Mobile.de}
\todo{Surname of Olli, Thomas Buenger knows?}
The eBay subsidiary Mobile.de is located in Berlin/Dreilinden and Germany's biggest online marketplace for vehicles. Ralf is the head of technology and was responsible for migrating the web platform from Perl to eBay's prevalent programming language Java in 2007. Andr\'e, Holger and OlliXXXXXXX are team leads of the consumer, commercial and mobile segment respectively.

They all mentioned the agile development style as being crucial for their cross-functional teams. During the last years of Scrum methodology and especially Kanban-like processes they were able to replace their weekly release cycles with multiple rollouts per day. They stated that this increased velocity helps them to fix bugs usually within a day.

From an architectural viewpoint, one of the biggest problems they mentioned was implementing internationalization in their platform. Characteristics of the different countries have to be considered, but due to their shared database, changes for one country often result in unintended side effects for others.

\subsection{XXX from Alacris Theranostics} Alacris is a small german company developing new approaches for the treatment of cancer. We spoke to XXX, who is part of the software development team. Her biggest problem is that she is working on a software system of which none of the original developers is in the team. She mentioned that code quality is a major problem, as well a non-existing test suite.

\subsubsection{XXXX from VMS} At VMS we spoke to XXX \todo{name}, who is a software engineer working on database integrations. During the interview he mentioned that working on real user data is superior to generated test data. Also he mentioned that he spends a lot of time debugging code, but often feels a lack of tool support for efficient debugging. This is why he often falls back to trial-and-error debugging.

\subsubsection{Summary} We interviewed a wide variety of programmers with very different problems. Especially programmers in big companies seemed to have problems that were not of technical but rather organizational nature. We decided it is not possible for us to work on these problems in the context of our project.

Even though the interviewees were very heterogenious, we were able to gain some insights:

\paragraph{Efficient tools are key to successful development} All developers we spoke to had a big interest in the tools they use. All were passionate about their tooling. They customize their tools themselves, add plug-ins and tinker with the configuration. Some build and maintain tools themselves. One company even had a dedicated group

When asked about the impact of tools on their development process, we heard strong opinions from describing how disfuntional tools limit the productivity of the developer up to that the well-functioning tools of the company are a major competitive advantage.

\paragraph{Programmers work best when in flow} Many programmers told us about their need for uninterrupted work, the so-called flow when they fully emerge their mind in the task at hand. But in their day-to-day work, interruptions seem to be rather the norm than the exception. The threshold over what constitutes as an interruption ranged from small things like switching windows over waiting longer times e.g. for a test run to complete to complete removel from the coding situation e.g. by being attending a meeting.
