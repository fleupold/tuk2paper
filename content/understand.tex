%!TEX root = ../document.tex
\section{Understand}
\label{sec:understand}
In the first phase of our research, we interviewed various programmers. In this section we will summarize the interviews and the insights we gained from them.

% TODO: Programmers name?
\paragraph{XXXXX from Psipenta}
Psipenta builds an ERP system for SMBs with a focus on manufactoring and suppy chain management. Their core product was developed initially in Cobol and parts have later been and still being ported to C++ and Java. Development happens largely in Eclipse, which has been largly extended and customized by Psipenta with automated code generation and integration into their workflow and environment. Work is flexible with home and remote work.

Our interview partner was XXXX. He is working on porting parts of the old code from Cobol to Java as well as developing entirely new features.

XXXXX estimates the time spent in daily communication at around 30\%. This includes answering emails and attending meetings with coleagues.

In the interview XXXX put an emphasis on code quality. A core means to achieve this was automated testing, best even with involving the customer.

As major churns, XXX mentioned interruptions, from co-workers as well as from a slow and/or unintuitive tool chain. As example he mentioned that building a running version of his product to manually test it takes more than half an hour.

\paragraph{Nicole von Steht from SAP}
SAP is one of the biggest enterprise application companies in the world. Nicole works as Frontend application developer at SAP. Her team consists of 12 people and uses the Scrum methodology. The team is located on one floor and most communication happens face-to-face.

Code quality is a core value, that is assured by unit and integration testing with continuous integration as well as rotating pair programming. Nicole writes code in a customized version of Eclipse.

\paragraph{Tillman Giese from SAP}
Tillman is working at SAP on the HANA Security Engine. His team consists of 8 people and is split over 2 offices.

Tillman regularly deals with other teams globally distributed, which is often difficult because of time zone differences. He is getting a lot of support requests, which he said most of the time are answered questions that the rquester could have been able to resolve themselves, if they would have just put in effort to research the question correctly. Often, the requests interrupt his workflow. To illustrate this state, Tillman said: \'\emph{I wish I could work on one task for a full hour}\'.

\paragraph{Christof Dorner from Readmill}
Readmill is a Berlin-based startup that builds a reader and social network for books. Christof is Backend Developer and works in a development team of 4.



As a side project, Tillman is running Agile development workshops at SAP.
