%!TEX root = ../document.tex

During our interviews, we briefly observed Christof Dorner from Readmill during his work. He showed us how he develops a feature by choosing a task from the bug and feature tracker Trello, creating a git branch locally, fixing the problem, writing tests for it and eventually opening a Pull Request on Github with the finished code to have it deployed later. Here we gained the insight, that a friction-less and uninterrupted flow is important. This can be achieved by customizing and constantly improving the used tools. Christof mentioned how he improved the flow over time, e.g. making tests run instantly on each change or working with a decentralized source code management tool.

We also observed the two current Bachelor projects from the HPI EPIC chair. Both bachelor projects work with real customer data as well as existing legacy code. During observation we noticed that one problem they face is a disconnect between understanding the code in the editor and the data in the database. This resulted in switching back and forth between the code editor and the HANA database studio as well as guessing values for variables in the queries.

Additionally to the interviews and observations, we took or own experience of developing applications with databases into account as well as incorperated experiences from the \emph{Enterprise Application Programming Model Research 2012} seminar 2012 at the HPI EPIC chair~\footnote{\url{https://github.com/lauritzthamsen/infrarecord/}}.

Eventually we gained our third insight, that also became the leading insight for the rest of the project:

\begin{quote}
\emph{Code needs data context}
\end{quote}

During the seminar we learned that not only any data context is needed but real customer data is important to make dependable predictions on performance and also well educated decisions about what matters and what does not.