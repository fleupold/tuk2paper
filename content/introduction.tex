%!TEX root = ../document.tex

\section[Introduction (Author: Patrick Schilf)]{Introduction}
Data sources explode in their size, complexity and speed they grow with. Whether Enterprise applications, games, or the vast amount of online services at least since the rise of mobile platforms represent a considerable percentage of the amount of data. Data is the central element that everything else revolves around. However, one has to understand that the question is no longer one of storage but rather one of processing these huge amounts. With databases growing not only in their total amount of tuples but also in the number of table columns, calculations become more complex and time critical. The introduction of SAP's \emph{HANA} marked a big step in addressing these problems. 

However, while it seems natural that software applications are tightly connected to their underlying data sources, this does surprisingly not apply to the actual development of these applications. IDEs such as Eclipse have come to be highly customizable tool chains that allow for integration of multiple tasks a software developer comes across in his daily work cycle. One that is missing here, however, is a linkage of code to its Data Context at the time of coding. In a conventional development workflow a programmer writes code, builds the application and tests it. Only in the last stage does he or she get in touch with the actual data that their applications are supposed to deal with, which means that potential errors or performance bottlenecks will only be discovered then. Development speed could be considerably increased if the programmer could take care of the mentioned issues from the very beginning in the cycle.

A second factor that interferes with developer's efficiency is that of the absence of a seamless workflow. There are bigger and smaller sources for interruptions a developer might face while coding. However, in many cases short interruptions tend to appear far more often than greater ones and can sum up to considerably distract from the actual problem a programmer is trying to solve. In the described situation a programmer does not only face the absence of a missing linkage between the code and the corresponding data, but he or she also has to navigate between multiple environments to compensate for this issue. Environments are often heterogenous, which even prevents an easy and comfortable transfer of code amongst them.

As part of the seminar \emph{"Trends and Concepts in the Software Industry II - Development of Enterprise Software"} held at the \emph{Hasso-Plattner-Institute} in Potsdam we developed a prototype to address the described inconveniences of a interruptive workflows as well as missing code-data-connections. This document presents our final iteration of the prototype as well as documents the design thinking inspired process we used to come up with our solution. Chapter \ref{sec:DESIGN_THINKING} lays down the process design thinking is based on, which also represents the guideline through this document. The following chapters reveal information we collected during interviews with people in different positions in the software industry and how they helped us to iteratively build our prototype. At the end of this report we suggest factors we believe need to be concerned to transform our concept into a working implementation.